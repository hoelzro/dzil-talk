\begin{frame}
\begin{center}
\LARGE{What is Dist::Zilla?}
\end{center}
\end{frame}

\begin{frame}
"Dist::Zilla builds distributions of code to be uploaded to the CPAN." \\
\end{frame}

\begin{frame}
Dist::Zilla goes a step further than other tools like MakeMaker.  It can generate MakeMaker input for you, among
many, many other things:

\begin{itemize}
\pause \item Seeds your distribution with common tests.
\pause \item Generates a README.
\pause \item Helps you to maintain a Changes file.
\pause \item Manages your MANIFEST and LICENSE files.
\pause \item Tags and pushes releases to your revision control system.
\pause \item Manages your META.yml/META.json for you.
\pause \item Manages your distribution's prerequisites.
\pause \item Uploads your release to CPAN.
\pause \item Weaves common POD into your modules.
\end{itemize}

\pause 
And much, much more!
\end{frame}

% To show off Dist::Zilla's file munging features...
\begin{frame}
\begin{center}
\LARGE{Before}
\end{center}
\end{frame}

\tiny{
\begin{minted}{perl}
package Foo::Bar;

use strict;
use warnings;

our $VERSION = '0.01';

1;

__END__

=head1 NAME

Foo::Bar -

=head1 SYNOPSIS

=head1 DESCRIPTION

=head1 METHODS

=head1 BUG REPORTING

...

=head1 COPYING

...

=head1 AUTHOR

Rob Hoelz <rob@hoelz.ro>

=head1 SEE ALSO

L<Foo>
\end{minted}
}

\newpage

\begin{frame}
\begin{center}
\LARGE{After}
\end{center}
\end{frame}
\tiny{
\begin{minted}{perl}
package Foo::Bar;

use strict;
use warnings;

1;

__END__

# ABSTRACT:

=head1 SYNOPSIS

=head1 DESCRIPTION

=head1 METHODS

=cut
\end{minted}
}
