\begin{frame}[fragile]
\begin{center}
\LARGE{Writing a Dist::Zilla Plugin}
\end{center}
\begin{shaded}
\begin{minted}{perl}
say "Moosey Fate";
\end{minted}
\end{shaded}
\end{frame}

\commentary{Dist::Zilla is written using Moose, so to write a plugin, you create a Moose class that consumes one or more roles.}
\commentary{Dist::Zilla does almost everything in terms of plugins; commands, the user interface, dist minting, dist building, etc.}
\commentary{Long story short: there are a \emph{lot} of roles in Dist::Zilla.  We're only going to cover a handful here.}
\commentary{Each role is based off of Dist::Zilla::Role::Plugin.}
\commentary{If you want to implement a plugin bundle, you can consume the Dist::Zilla::Role::PluginBundle role.}

\seperatorslide{The Dist::Zilla Lifecycle}
\begin{itemslide}{Plugin Order for Build}
\item BeforeBuild
\item FileGatherer
\item FilePruner
\item FileMunger
\item PrereqSource
\item MetaProvider
\item InstallTool
\item AfterBuild
\end{itemslide}

\begin{frame}
In addition to the previous plugins being run, the \emph{test} and \emph{release} commands
have their own lifecycle roles, namely TestRunner in the case of the former, and BeforeRelease, Releaser, and AfterRelease in
the case of the latter.
\end{frame}

\seperatorslide{RemoveUnsightlyComments Plugin}

\begin{frame}[fragile]
\begin{shaded}
\begin{minted}{ini}
[RemoveUnsightlyComments]
\end{minted}
\end{shaded}
\end{frame}

\begin{frame}[fragile]
\begin{shaded}
\begin{minted}{ini}
[RemoveUnsightlyComments]
marker = XXX
marker = FOR ROB
\end{minted}
\end{shaded}
\end{frame}

\inputexample{Dist-Zilla-Plugin-RemoveUnsightlyComments/corpus/lib/Fake.pm}
\inputexample{Dist-Zilla-Plugin-RemoveUnsightlyComments/t/01-basic.t}
\inputexample{Dist-Zilla-Plugin-RemoveUnsightlyComments/t/02-options.t}
\commentary{We're not going to go over dist.ini here, since it's basically a copy of the one from the earlier example.}
\inputexample{Dist-Zilla-Plugin-RemoveUnsightlyComments/lib/Dist/Zilla/Plugin/RemoveUnsightlyComments.pm}
