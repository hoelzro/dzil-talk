\begin{frame}
Now that we've covered the basic command line usage, let's take a look at a config file.

\pause

Dist::Zilla's config file is located at the root of your distribution's directory hierarchy, and
is named \textit{dist.ini}.
\end{frame}

\begin{frame}
\begin{shaded}
name              = Rob-DZil-ExampleOne \newline
author            = Rob Hoelz $<$rob@hoelz.ro$>$ \newline
license           = Perl\_5 \newline
copyright\_holder = Rob Hoelz \newline
copyright\_year   = 2011 \newline
version           = 0.01 \newline

[@Basic]
\end{shaded}
\end{frame}

\begin{frame}
Wow, that was easy!

% explain that [@Basic] is a plugin bundle

\pause
"[@Basic]" is what's known as a \textit{plugin bundle}.  It's a nice way to refer to a bunch of plugins
at once.  For example, "[@Basic]" is equivalent to:
\end{frame}

\begin{frame}
\begin{shaded}
[GatherDir] \newline
[PruneCruft] \newline
[ManifestSkip] \newline
[MetaYAML] \newline
[License] \newline
[Readme] \newline
[ExtraTests] \newline
[ExecDir] \newline
[ShareDir] \newline
[MakeMaker] \newline
[Manifest] \newline
[TestRelease] \newline
[ConfirmRelease] \newline
[UploadToCPAN]
\end{shaded}
\end{frame}
