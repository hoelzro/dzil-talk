\begin{frame}
\begin{center}
\LARGE{Example Distribution}
\end{center}
\end{frame}

\begin{frame}
\begin{shaded}
dzil new Rob::DZil::Example
\end{shaded}
\end{frame}

\begin{frame}
\begin{shaded}
|-- Changes
|-- dist.ini
|-- lib
|   `-- Rob
|       `-- DZil
|           `-- Example.pm
`-- weaver.ini
\end{shaded}
\end{frame}

\begin{frame}
\begin{shaded}
\inputminted{perl}{Rob-DZil-Example/t/basic.t}
\end{shaded}
\end{frame}

\begin{frame}
\begin{shaded}
\tiny{\inputminted{perl}{Rob-DZil-Example/lib/Rob/DZil/Example.pm}}
\end{shaded}
\end{frame}

\begin{frame}
\begin{shaded}
\tiny{\inputminted{perl}{Rob-DZil-Example/dist.ini}}
\end{shaded}
\end{frame}

\begin{frame}
And now...\pause we build!

\begin{shaded}
dzil build
\end{shaded}
\end{frame}

\begin{frame}
This will build a distribution tarball, along with an unpacked directory
with its contents.  Those contents are...
\end{frame}

\begin{verbatim}
|-- Changes
|-- LICENSE
|-- MANIFEST
|-- META.yml
|-- Makefile.PL
|-- README
|-- lib
|   `-- Rob
|       `-- DZil
|           `-- Example.pm
`-- t
    `-- basic.t
\end{verbatim}
